\documentclass[../report.tex]{subfiles}
\begin{document}
\section{Building and Running the Application}
\subsection{Building}

	This step is not strictly necessary to run the application as it is available as pre-built docker images but is included for completeness.  To build (probably) requires a Unix like environment (e.g. Linux or Mac OS), Python 3.5+ with pip, GNU make and docker.
	
\begin{verbatim}
	$ cd seismic-detector
	$ python3 -m virtualenv .
	$ source bin/activate
	$ pip install numpy
	$ make clean sdist docker
\end{verbatim}

\subsection{Running}

	To run the application pre-built requires docker and docker-compose.  It is hosted on the Docker Hub so it does not need to be built first.
	
	From the root of the project (on writeable media such as a hard disk), invoke \texttt{docker-compose up -d}.
	
\begin{verbatim}
	$ docker-compose up -d
	docker-compose up -d
	Creating network "seismic_default" with the default driver
	Creating seismic_api_1
	Creating seismic_minio_1
	Creating seismic_postgres_1
	Creating seismic_redis_1
	Creating seismic_worker_1
	Creating seismic_interface_1
\end{verbatim}

	The application should now be available via a browser on \texttt{http://localhost:8080}.  To terminate, call \texttt{docker-compose down}.
\end{document}
\documentclass[../report.tex]{subfiles}

\begin{document}
	\subsection{Seismic Waves}
	\label{sec:seismicwaves}
	Seismic waves take on two main forms, body waves and surface waves.  Body waves are those that travel through the interior of the earth and are the fastest travelling.  The body waves are comprised of \textbf{P} (primary) waves which are compressional waves, travel fastest and thus arrive first.  \textbf{S} (secondary) waves are shear waves and travel more slowly, thus arrive later.  The separation between the phases is related to the distance of the earthquake and the local velocity structure. The surface waves travel only along the earth's crust and, as they are confined to shallow depths where seismic velocities are slow, will normally arrive much later than the body waves.
	
	A seismic station records movement over three axis: vertical (\textbf{z}) alongside horizontal in terms of north-south (\textbf{n}) and east-west (\textbf{e}).  Due to seismic velocities generally increasing with depth, P waves arrive at a seismic station close to the vertical axis.  As a result, P waves can be measured as a simple metric of displacement along the Z axis.  S waves follow similar ray paths, but have their particle motion perpendicular to the direction of propagation.  As a result, they manifest on a seismogram as movement on both the n and e axis. The geometry of the fault tends to have a bearing on the orientation of the displacement so the two horizontal axis of movement cannot be easily combined in to a single metric for time series analysis.
\end{document}
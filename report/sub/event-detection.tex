\documentclass[../report.tex]{subfiles}
 
	The initial idea of the project was to be able to use the \textit{SAX} data to detect events in a measurement over a long period of time.  This ultimately proved to be too difficult to achieve due to the normalisation problems mentioned in \cref{sec:Z-normalisation}.
	
	As such, an algorithm was needed to achieve this on unprocessed data.
	
\subsection{Algorithm}
	The algorithm works by iterating through the data on two sliding windows. calculating a short and long term mean of the absolute values of the amplitude.

	
	\noindent
	\\
	$l = $ number of datapoints in LTA \\
	$s = $ number of datapoints in STA

\begin{equation}
	\overline{LTA} = \sum_{i=1}^{l}\dfrac{\abs{v(t - i)}}{l}
\end{equation}

\begin{equation}
	\overline{STA} = \sum_{i=1}^{s}\dfrac{\abs{v(t - i)}}{s}
\end{equation}
	
	For each iteration, the STA is compared to a number of standard deviations (typically 3) from the LTA, if it exceeds this value, the event is considered triggered on.  The current value of the LTA is then stored.
	
	The iteration then continues calculating STA with the triggered value set to true until the STA drops below the original LTA value when the event was triggered.
	
	If the event duration is above a pre-set threshold (typically 5s) then the event is recorded, if not the it is discarded as a probable spike.	

	The downside to this approach is that it cannot normally detect an event within the period of the LTA from the start of the observation window.

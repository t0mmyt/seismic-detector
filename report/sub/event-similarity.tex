\documentclass[../report.tex]{subfiles}

\subsection{Bag of Words} \label{sec:bow}
	Many text similarity models rely on having a collection of words to be treated as features.  Often delimiting text by white-space, removing punctuation and normalising to a single case.  The output strings from the SAX algorithm have none of these features and are effectively a single long word so this approach of splitting isn't appropriate
	
	One approach to this is to take a sliding window of length \textit{w} to build to vocabulary \citep{sax-hot}.  This was implemented in Python as follows:

\lstinputlisting[language=Python]{../seismic/similarity/bag_of_words.py}

	For example, given an input string \textit{s} of "abcdefghijkl" and a word length \textit{w} of 5, the function returns the list: 
\begin{verbatim}
	['abdce', 'bdcef', 'dcefg', 'cefgh', 'efghi', 'fghij', 'ghijk', 'hijkl']
\end{verbatim}

	Test cases for this function are included in \cref{sec:bow-test}.


\subsection{Jaccard Similarity} \label{sec:jaccard}
\subsubsection{Implementation}
	The Jaccard Similarity Coefficient was implemented in Python as follows (with the case two sets with a cardinality of 0 returning 1).

\lstinputlisting[language=Python]{../seismic/similarity/jaccard.py}

	Test cases are included in \cref{sec:jaccard-test}.  The inputs to the function are the bag-of-words produced from two observations.
	
\subsubsection{Results}

	The Jaccard similarity was run against 138 events with known similarities from the Nabro 2011 data for 4 stations (NAB1..4).  For each event at a given station, a SAX string was produced with the length of the alphabet and the PAA interval as varying parameters between runs.  This was then run through the bag\_of\_words function with varying word lengths.  In differing runs, the PAA Interval was set to 5, 10, 25 and 50ms, the alphabet was varied from a length of 4 to 9 characters and the length of the words were set to 5 to 10.
	
	For each run, a similarity matrix was produced comparing every pair of events and the similarities ranked.  This was then aggregated in to a single data frame and the ranks and average ranks compared to whether this was a known similar event.  The results for the following parameters are shown below (these parameters showed the best case for matching known events):
	
\begin{verbatim}
paa_int = 10
alphabet = "abcde"
word_len = 5
\end{verbatim}

\pgfplotstabletypeset[%
	col sep=comma,
	string type,
	every head row/.style={
		before row={
			\toprule
			& \multicolumn{5}{c}{Jaccard Index} & \multicolumn{5}{c}{Rank in Cohort}\\
		},
		after row=\midrule,
	},
	every last row/.style={after row=\bottomrule},
	columns/MeanScore/.style={column name={Mean}},
	columns/NAB1-Rank/.style={column name={NAB1}},
	columns/NAB2-Rank/.style={column name={NAB2}},
	columns/NAB3-Rank/.style={column name={NAB3}},
	columns/NAB4-Rank/.style={column name={NAB4}},
	columns/MeanRank/.style={column name={Mean}},
]{data/jaccard.csv}

	The table shows the 20 most similar pairs of events sorted by their average rank across the four stations.  The \textit{Match} column shows whether or not the event was a known match.  As can be seen, while some of the matched events score quite highly, there are also many highly scored false positives.  The rankings are of a maximum of 137.  The mean rank for a known match was $48.8$ ($\sigma = 24.7$) and a known non-match was $69.3$ ($\sigma = 23.7$) of 137.  While matched events were more likely to be considered of higher similarity using this method, the effect is only marginal and therefore not a suitable predictor by itself.

\subsection{Vector Space Model (TF-IDF)}
\subsubsection{Implementation}
\subsubsection{Results}
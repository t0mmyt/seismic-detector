\documentclass[../report.tex]{subfiles}
\begin{document}

\subsection{Abstract}

	The purpose of this project is to develop an infrastructure and tool set for converting raw seismic time series data into a searchable string using SAX (\textbf{S}ymbolic \textbf{A}ggregate appro\textbf{X}imation) and then to store this data as a suffix tree for fast searching and analysis.  An interface will then be developed to enable the searching of these suffix trees and provide visualisation of the data.  Primarily this will be with the aim of being able to identify the start of an event with reasonable accuracy.  Additionally it could ideally be used to search for similar patterns over time or between stations after an event.
	
\subsection{Objectives} \label{sec:objectives}
	\begin{enumerate}
		\item To facilitate the importing and storage of raw seismic data and associated meta-data
		\item To calculate and store SAX strings of a whole observation period or an event
		\item To be able to determine the onset of an event in an observation
		\item To provide graphical representations of the analysis
		\item To provide a web based user interface for all of the above
		\item To evaluate the suitability of SAX or SAX-VSM for comparing the patterns of seismic events.
	\end{enumerate}

	\textit{These objectives are slightly altered from the original proposal.  This is discussed in \cref{sec:alteration-objectives}}.
		
\end{document}
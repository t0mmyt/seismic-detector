\documentclass[../report.tex]{subfiles}
\begin{document}

\subsection{Event Detection}

	The SAX detection algorithm performed quite well on an observation with a single event, often within a few milliseconds of the known onset.  Applied naively to a longer time period with multiple events it was much less reliable.  When combined with the distribution function its performance was fair, it detected many events but still missed quite a few.
	
	There are not sufficient patterns in the produced SAX string beyond the change in character distribution that can be used to determine when an event is happening.
	
	While the analysis of the strings is fast, the pre-computation during the PAA phase (the normalisation and aggregation) must still be factored in with the consideration that the string is only to be used once and then discarded as the PAA needs to be recalculated once the event has been isolated.
	
	In conclusion, it is only performing a step change algorithm with the steps being predefined by the Gaussian break points as determined by the SAX algorithm.  It is not felt that it has anything to add over existing step change based algorithms such as \textit{Short Term Average Long Term Average} (STALTA) already in use in the field.


\subsection{Event Similarity}

	Ultimately SAX was found not to be suitable for detecting similarities between known similar seismic events.  This is presumably down to the factors listed in \cref{sec:seismicwaves}, specifically the path effect.  That is the reflection and refraction of the waves as they pass through different substrates and geological boundaries causes a significant amount of noise that distorts the waveform beyond it being easily recognisable.  While the string similarity measures (Jaccard and TF-IDF) showed on average a slightly higher score for known matched events, the highly scoring false positives rendered the positive results statistically insignificant.
	
\end{document}